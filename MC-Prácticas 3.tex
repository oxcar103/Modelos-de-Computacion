%%
% Modificación de una plantilla de Latex para adaptarla al castellano.
%%

%%%%%%%%%%%%%%%%%%%%%
% Thin Sectioned Essay
% LaTeX Template
% Version 1.0 (3/8/13)
%
% This template has been downloaded from:
% http://www.LaTeXTemplates.com
%
% Original Author:
% Nicolas Diaz (nsdiaz@uc.cl) with extensive modifications by:
% Vel (vel@latextemplates.com)
%
% License:
% CC BY-NC-SA 3.0 (http://creativecommons.org/licenses/by-nc-sa/3.0/)
%
%%%%%%%%%%%%%%%%%%%%%

%----------------------------------------------------------------------------------------
%	PACKAGES AND OTHER DOCUMENT CONFIGURATIONS
%----------------------------------------------------------------------------------------

\documentclass[a4paper, 11pt]{article} % Font size (can be 10pt, 11pt or 12pt) and paper size (remove a4paper for US letter paper)

\usepackage[protrusion=true,expansion=true]{microtype} % Better typography
\usepackage[utf8]{inputenc}
\usepackage{enumerate}
\usepackage{enumitem}

% sudo apt-get install texlive-lang-spanish
\usepackage[spanish]{babel} % English language/hyphenation
\selectlanguage{spanish}
% Hay que pelearse con babel-spanish para el alineamiento del punto decimal
\decimalpoint

% Símbolos matemáticos
\usepackage{amssymb}
\let\oldemptyset\emptyset
\let\emptyset\varnothing

\usepackage{tikz}
\usetikzlibrary{automata,positioning,babel,calc}

\makeatletter
\renewcommand{\maketitle}{ % Customize the title - do not edit title and author name here, see the TITLE block below
\begin{flushright} % Right align
{\LARGE\@title} % Increase the font size of the title

\vspace{50pt} % Some vertical space between the title and author name

{\large\@author} % Author name
\\\@date % Date

\vspace{40pt} % Some vertical space between the author block and abstract
\end{flushright}
}


%----------------------------------------------------------------------------------------
%	TITLE
%----------------------------------------------------------------------------------------

\title{\textbf{Modelos de Computación}\\ % Title
Relación de Prácticas 3} % Subtitle

\author{\textsc{Óscar Bermúdez} % Author
\\{\textit{Universidad de Granada}}} % Institution

\date{\today} % Date

%----------------------------------------------------------------------------------------
\begin{document}

\maketitle % Print the title section

\begin{enumerate}
	\item Obtener un AFD capaz de aceptar las cadenas $u \in \{0,1\}^*$, que contengan simultánemente las cadenas $000$ y $111$.
	\begin{center}
		\begin{tikzpicture}[>=stealth,auto]
			\node[state,initial] (q_0)   {};
			\node[state] (q_1) [above right=of q_0] {};
			\node[state] (q_2) [below right=of q_0] {};
			\node[state] (q_3) [above right=of q_1] {};
			\node[state] (q_4) [below right=of q_2] {};
			\node[state] (q_5) [right=of q_3] {};
			\node[state] (q_6) [right=of q_4] {};
			\node[state] (q_7) [right=of q_5] {};
			\node[state] (q_8) [right=of q_6] {};
			\node[state] (q_9) [below right=of q_7] {};
			\node[state] (q_10) [above right=of q_8] {};
			\node[state,accepting](q_11) [above right=of q_10] {};
			\node[state] (q_12) at ($(q_0)!0.5!(q_11)$) {};
			\path[->]
			(q_0) edge  node {0} (q_1)
			(q_0) edge  [swap] node {1} (q_2)
			(q_1) edge  node {0} (q_3)
			(q_1) edge  node {1} (q_12)
			(q_2) edge  node {0} (q_12)
			(q_2) edge  [swap] node {1} (q_4)
			(q_3) edge  node {0} (q_5)
			(q_3) edge  node {1} (q_12)
			(q_4) edge  node {0} (q_12)
			(q_4) edge  [swap] node {1} (q_6)
			(q_5) edge  node {0} (q_12)
			(q_5) edge  node {1} (q_7)
			(q_6) edge  [swap] node {0} (q_8)
			(q_6) edge  node {1} (q_12)
			(q_7) edge  node {0} (q_12)
			(q_7) edge  node {1} (q_9)
			(q_8) edge  [swap] node {0} (q_10)
			(q_8) edge  node {1} (q_12)
			(q_9) edge  node {0} (q_12)
			(q_9) edge  node {1} (q_11)
			(q_10) edge  [swap] node {0} (q_11)
			(q_10) edge  node {1} (q_12)
			(q_11) edge [bend left=60] node {0} (q_8)
			(q_11) [swap] edge [bend right=60] node {1} (q_7)
			(q_12) edge [loop left]  node {0,1} (q_12);
		\end{tikzpicture}
	\end{center}
	\pagebreak
	\item Obtener un AFD equivalente al siguiente AFND:
	\begin{center}
		\begin{tikzpicture}[>=stealth,auto]
			\node[state,initial] (q_1)   {$q_1$};
			\node[state] (q_2) [right=of q_1] {$q_2$};
			\node[state] (q_3) [right=of q_2] {$q_3$};
			\node[state,accepting] (q_4) [right=of q_3] {$q_4$};
			\path[->]
			(q_1) edge  [loop above] node {0,1} (q_1)
			(q_1) edge  node {1} (q_2)
			(q_2) edge  node {0, $\varepsilon$} (q_3)
			(q_3) edge  node {0,1} (q_4)
			(q_4) edge  [loop above] node {0,1} (q_4);
		\end{tikzpicture}
	\end{center}
	
	\begin{center}
		\begin{tikzpicture}[>=stealth,auto]
			\node[state,initial] (q_0)   {$\{q_1\}$};
			\node[state] (q_1) [right=of q_0] {$\{q_1,q_2,q_3\}$};
			\node[state,accepting] (q_2) [below right=of q_1] {$\{q_1,q_3,q_4\}$};
			\node[state,accepting] (q_3) [above right=of q_1] {$\{q_1,q_2,q_3,q_4\}$};
			\node[state,accepting] (q_4) [above right=of q_2] {$\{q_1,q_4\}$};
			\path[->]
			(q_0) edge  [loop above] node {0} (q_1)
			(q_0) edge  node {1} (q_1)
			(q_1) edge  node {0} (q_2)
			(q_1) edge  node {1} (q_3)
			(q_2) edge  node {0} (q_4)
			(q_2) edge  [bend left=10] node {1} (q_3)
			(q_3) edge  [bend left=10] node {0} (q_2)
			(q_3) edge  [loop above] node {1} (q_3)
			(q_4) edge  [loop right] node {0} (q_3)
			(q_4) edge  node {1} (q_3);
		\end{tikzpicture}
	\end{center}
	\pagebreak
	\item \begin{enumerate}[label=\alph*)]
		\item Construir un AFND a partir de cada una de las siguientes expresiones regulares:
		\begin{itemize}
			\item $(aa)^*b^*$
			\begin{center}
				\begin{tikzpicture}[>=stealth,auto]
					\node[state,initial above,accepting] (q_0)   {$q_0$};
					\node[state] (q_1) [right=of q_0] {$q_1$};
					\node[state,accepting] (q_2) [left=of q_0] {$q_2$};
					\path[->]
					(q_0) edge  [bend left] node {a} (q_1)
					(q_0) edge  node {b} (q_2)
					(q_1) edge  [bend left] node {a} (q_0)
					(q_2) edge  [loop left]node {b} (q_2);
				\end{tikzpicture}
			\end{center}
			\item $a(b+a)^*b$
			\begin{center}
				\begin{tikzpicture}[>=stealth,auto]
					\node[state,initial] (q_0)   {$q_0$};
					\node[state] (q_1) [right=of q_0] {$q_1$};
					\node[state,accepting] (q_2) [right=of q_1] {$q_2$};
					\path[->]
					(q_0) edge  node {a} (q_1)
					(q_1) edge  node {b} (q_2)
					(q_1) edge  [loop above] node {a,b} (q_1);
				\end{tikzpicture}
			\end{center}
		\end{itemize}
		\item Transformar los AFND's obtenidos en el apartado anterior a AFD's.
		\begin{itemize}
			\item $(aa)^*b^*$
			\begin{center}
				\begin{tikzpicture}[>=stealth,auto]
					\node[state,initial above,accepting] (q_0)   {$\{q_0\}$};
					\node[state] (q_1) [right=of q_0] {$\{q_1\}$};
					\node[state,accepting] (q_2) [left=of q_0] {$\{q_2\}$};
					\node[state] (q_3) [below=of q_0] {$\{\}$};
					\path[->]
					(q_0) edge  [bend left] node {a} (q_1)
					(q_0) edge  [swap] node {b} (q_2)
					(q_1) [swap] edge  [bend left] node {a} (q_0)
					(q_1) edge  [swap] node {b} (q_3)
					(q_2) edge  [loop left] node {b} (q_2)
					(q_2) edge  node {a} (q_3)
					(q_3) edge  [loop below]node {a,b} (q_3);
				\end{tikzpicture}
			\end{center}
			\item $a(b+a)^*b$
			\begin{center}
				\begin{tikzpicture}[>=stealth,auto]
					\node[state,initial below] (q_0)   {$\{q_0\}$};
					\node[state] (q_1) [right=of q_0] {$\{q_1\}$};
					\node[state,accepting] (q_2) [right=of q_1] {$\{q_1,q_2\}$};
					\node[state] (q_3) [left=of q_0] {$\{\}$};
					\path[->]
					(q_0) edge  node {a} (q_1)
					(q_0) edge  [swap] node {b} (q_3)
					(q_1) edge  [loop above] node {a} (q_1)
					(q_1) edge  [bend left] node {b} (q_2)
					(q_2) edge  [bend left] node {a} (q_1)
					(q_2) edge  [loop right] node {b} (q_2)
					(q_3) edge  [loop left] node {a,b} (q_3);
				\end{tikzpicture}
			\end{center}
		\end{itemize}
	\end{enumerate}
\end{enumerate}
\end{document}
