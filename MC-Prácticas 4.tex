%%
% Plantilla de Memoria
% Modificación de una plantilla de Latex de Nicolas Diaz para adaptarla 
% al castellano y a las necesidades de escribir informática y matemáticas.
%
% Editada por: Mario Román
%
% License:
% CC BY-NC-SA 3.0 (http://creativecommons.org/licenses/by-nc-sa/3.0/)
%%

%%%%%%%%%%%%%%%%%%%%%
% Thin Sectioned Essay
% LaTeX Template
% Version 1.0 (3/8/13)
%
% This template has been downloaded from:
% http://www.LaTeXTemplates.com
%
% Original Author:
% Nicolas Diaz (nsdiaz@uc.cl) with extensive modifications by:
% Vel (vel@latextemplates.com)
%
% License:
% CC BY-NC-SA 3.0 (http://creativecommons.org/licenses/by-nc-sa/3.0/)
%
%%%%%%%%%%%%%%%%%%%%%

%----------------------------------------------------------------------------------------
%	PAQUETES Y CONFIGURACIÓN DEL DOCUMENTO
%----------------------------------------------------------------------------------------

%% Configuración del papel.
% microtype: Tipografía.
% mathpazo: Usa la fuente Palatino.
\documentclass[a4paper, 11pt]{article}
\usepackage[protrusion=true,expansion=true]{microtype}
\usepackage{mathpazo}

% Indentación de párrafos para Palatino
\setlength{\parindent}{0pt}
  \parskip=8pt
\linespread{1.05} % Change line spacing here, Palatino benefits from a slight increase by default


%% Castellano.
% noquoting: Permite uso de comillas no españolas.
% lcroman: Permite la enumeración con numerales romanos en minúscula.
% fontenc: Usa la fuente completa para que pueda copiarse correctamente del pdf.
\usepackage[spanish,es-noquoting,es-lcroman]{babel}
\usepackage[utf8]{inputenc}
\usepackage[T1]{fontenc}
\selectlanguage{spanish}


%% Gráficos
\usepackage{graphicx} % Required for including pictures
\usepackage{wrapfig} % Allows in-line images
\usepackage[usenames,dvipsnames]{color} % Coloring code


%% Matemáticas
\usepackage{amsmath}


%% Bibliografía
\makeatletter
\renewcommand\@biblabel[1]{\textbf{#1.}} % Change the square brackets for each bibliography item from '[1]' to '1.'
\renewcommand{\@listI}{\itemsep=0pt} % Reduce the space between items in the itemize and enumerate environments and the bibliography

\usepackage{tikz}
\usetikzlibrary{automata,positioning}

%----------------------------------------------------------------------------------------
%	TÍTULO
%----------------------------------------------------------------------------------------
% Configuraciones para el título.
% El título no debe editarse aquí.
\renewcommand{\maketitle}{
  \begin{flushright} % Right align
  
  {\LARGE\@title} % Increase the font size of the title
  
  \vspace{50pt} % Some vertical space between the title and author name
  
  {\large\@author} % Author name
  \\\@date % Date
  \vspace{40pt} % Some vertical space between the author block and abstract
  \end{flushright}
}

% Título
\title{\textbf{Modelos de Computación}\\ % Title
Relación de Prácticas 4} % Subtitle

\author{\textsc{Óscar Bermúdez Garrido} % Author
\\{\textit{Universidad de Granada}}} % Institution

\date{\today} % Date



%----------------------------------------------------------------------------------------
%	DOCUMENTO
%----------------------------------------------------------------------------------------

\begin{document}

\maketitle % Print the title section

\begin{enumerate}
	\item Dados los alfabetos $A=\{0,1,2,3\}$ y $B=\{0,1\}$ y el homomorfismo $f$ de $A^*$ a $B^*$ dado por: $f(0)=00$, $f(1)=01$, $f(2)=10$, $f(3)=11$. Resolver las siguientes cuestiones:
	
	\begin{enumerate}
		\item Sea $L_1$ el conjunto de palabras de $B^*$ tales que acaban con la subcadena $11$.\\
		Construir un autómata finito determinista que acepte $f^{-1}(L_1)$.
		
		$L_1=\{B^*11: B=\{0,1\}\} \Rightarrow f^{-1}(L_1)=\{A^*3: A=\{0,1,2,3\}\}$
		\begin{center}
			\begin{tikzpicture}[>=stealth,auto]
				\node[initial,state] (q_0) {};
				\node[state,accepting](q_1) [right=of q_0] {};
				\path[->]
				(q_0) edge  [loop above] node {0,1,2} (q_0)
				(q_0) edge  [bend left] node {3} (q_1)
				(q_1) edge  [bend left] node {0,1,2} (q_0)
				(q_1) edge  [loop above] node {3} (q_1);
			\end{tikzpicture}
		\end{center}
			
		\item Sea $L_3$ el conjunto de palabras de $A^*$ definido como $L_3=\{2^k3^k: 1 \leq k \leq 100 \}$.\\
		Construir una expresión regular que represente a $f(L_3)$.
		
		$L_3=\{2^k3^k: 1 \leq k \leq 100 \} \Rightarrow f(L_3)=\{(10)^k(11)^k: 1 \leq k \leq 100 \}$
		
		Sea $g: \mathcal{L} \rightarrow \mathcal{L}$ \footnote{$\mathcal{L}$ es el conjunto de expresiones regulares} una función que toma la expresión regular $r$ y devuelve la expresión regular $g(r)=\varepsilon + 10(r)11$.
		
		También, usaremos la siguente notación, $g^n(r) = g^{n-1}(g(r))$ con $g^1(r)=g(r)$.
		
		Ahora, es fácil ver que $f(L_3)$ se puede expresar como la expresión regular $10(g^{99}(\varepsilon))11$
		
	\end{enumerate}
	
	\item Construir un autómata finito determinista que acepte el lenguaje $L_2=\{0^i1^j: i \geq j \}$.\\
	Por lema de bombeo, podemos comprobar que $L_2$ no es un lenguaje regular y, por tanto, no podemos construir un autómata finito determinista asociado a dicho lenguaje.
	
	
	\item Minimizar si es posible el siguiente autómata usando el algoritmo visto en clase:
	\begin{center}
		\begin{tikzpicture}[>=stealth,auto]
			\node[state] (A)   {$A$};
			\node[state] (B) [above right=of A] {$B$};
			\node[initial,state,accepting](C) [below right=of A] {$C$};
			\node[state](D) [below right=of B] {$D$};
			\node[state,accepting](E) [right=of B] {$E$};
			\path[->]
			(A) edge  node {0,1} (B)
			(B) edge  [swap] node {0} (C)
			(B) edge  node {1} (E)
			(C) edge  [swap] node {0} (D)
			(C) edge  node {1} (A)
			(D) edge  node {0,1} (B)
			(E) edge  node {0,1} (D);
		\end{tikzpicture}
	\end{center}
	
	Por el algoritmo de identificación de estados indistiguibles, obtenemos que $A \equiv D$ y $C \equiv E$, por tanto, el autómata minimizado es:
	\begin{center}
		\begin{tikzpicture}[>=stealth,auto]
			\node[state] (A)   {$A,D$};
			\node[state] (B) [above right=of A] {$B$};
			\node[initial,state,accepting](C) [below right=of A] {$C,E$};
			\path[->]
			(A) edge  node {0,1} (B)
			(B) edge  node {0,1} (C)
			(C) edge  node {0,1} (A);
		\end{tikzpicture}
	\end{center}
	
\end{enumerate}

\end{document}
