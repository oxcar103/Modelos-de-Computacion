%%
% Modificación de una plantilla de Latex para adaptarla al castellano.
%%

%%%%%%%%%%%%%%%%%%%%%
% Thin Sectioned Essay
% LaTeX Template
% Version 1.0 (3/8/13)
%
% This template has been downloaded from:
% http://www.LaTeXTemplates.com
%
% Original Author:
% Nicolas Diaz (nsdiaz@uc.cl) with extensive modifications by:
% Vel (vel@latextemplates.com)
%
% License:
% CC BY-NC-SA 3.0 (http://creativecommons.org/licenses/by-nc-sa/3.0/)
%
%%%%%%%%%%%%%%%%%%%%%

%----------------------------------------------------------------------------------------
%	PACKAGES AND OTHER DOCUMENT CONFIGURATIONS
%----------------------------------------------------------------------------------------

\documentclass[a4paper, 11pt]{article} % Font size (can be 10pt, 11pt or 12pt) and paper size (remove a4paper for US letter paper)

\usepackage[protrusion=true,expansion=true]{microtype} % Better typography
\usepackage[utf8]{inputenc}
\usepackage{enumerate}
\usepackage{enumitem}

% sudo apt-get install texlive-lang-spanish
\usepackage[spanish]{babel} % English language/hyphenation
\selectlanguage{spanish}
% Hay que pelearse con babel-spanish para el alineamiento del punto decimal
\decimalpoint

% Símbolos matemáticos
\usepackage{amssymb}
\let\oldemptyset\emptyset
\let\emptyset\varnothing

\usepackage{tikz}
\usetikzlibrary{automata,positioning,babel}

\makeatletter
\renewcommand{\maketitle}{ % Customize the title - do not edit title and author name here, see the TITLE block below
\begin{flushright} % Right align
{\LARGE\@title} % Increase the font size of the title

\vspace{50pt} % Some vertical space between the title and author name

{\large\@author} % Author name
\\\@date % Date

\vspace{40pt} % Some vertical space between the author block and abstract
\end{flushright}
}


%----------------------------------------------------------------------------------------
%	TITLE
%----------------------------------------------------------------------------------------

\title{\textbf{Modelos de Computación}\\ % Title
Relación de Prácticas 2} % Subtitle

\author{\textsc{Óscar Bermúdez} % Author
\\{\textit{Universidad de Granada}}} % Institution

\date{\today} % Date

%----------------------------------------------------------------------------------------
\begin{document}

\maketitle % Print the title section

\begin{enumerate}
	\item Considere el siguiente $AFD$ $M=(Q, A, \partial, q_0, F)$, donde
	\begin{itemize}
		\item $Q =\{q_0, q_1, q_2\}$
		\item $A=\{0,1\}$
		\item La función de transición viene dada por:
		\begin{itemize}[label=]
			\item $\partial (q_0, 0) = q_1$
			\item $\partial (q_0, 1) = q_0$
			\item $\partial (q_1, 0) = q_2$
			\item $\partial (q_1, 1) = q_0$
			\item $\partial (q_2, 0) = q_2$
			\item $\partial (q_2, 1) = q_2$
		\end{itemize}
		\item $F=\{q_2\}$
	\end{itemize}
	Describa informalmente el lenguaje aceptado:\\
	
	Primero, dibujaremos el AFD para aclararnos qué significan las condiciones anteriores:
	\begin{center}
		\begin{tikzpicture}[>=stealth,auto]
			\node[state,initial] (q_0)   {$q_0$};
			\node[state] (q_1) [right=of q_0] {$q_1$};
			\node[state,accepting](q_2) [right=of q_1] {$q_2$};
			\path[->]
			(q_0) edge  [bend left] node {0} (q_1)
			(q_0) edge  [loop above] node {1} (q_0)
			(q_1) edge  node {0} (q_2)
			(q_1) edge  [bend left] node {1} (q_0)
			(q_2) edge  [loop above] node {0,1} (q_2);
		\end{tikzpicture}
	\end{center}
	Como podemos ver, una vez lleguemos a $q_2$, cualquier palabra será válida. Para llegar a $q_2$ basta encontrar la subcadena $00$.\\
	Por tanto, el lenguaje generado es $L=\{A^*00A^*\}$ con $A=\{0,1\}$.
	
	\item Dibujar los AFDs que aceptan los siguientes lenguajes con alfabeto $\{0,1\}$:
	\begin{enumerate}[label=\alph*)]
		\item El lenguaje vacío, $\emptyset$.
			\begin{center}
				\begin{tikzpicture}[>=stealth,auto]
					\node[state,initial] (q_0) {};
					\path[->]
					(q_0) edge  [loop right] node {0,1} (q_0);
				\end{tikzpicture}
			\end{center}
		\item El lenguaje formado por la palabra vacía, o sea, $\{\varepsilon\}$.
			\begin{center}
				\begin{tikzpicture}[auto]
					\node[state,initial,accepting] (q_0) {};
					\node[state] (q_1) [right=of q_0] {};
					\path[->]
					(q_0) edge  node {0,1} (q_1)
					(q_1) edge  [loop right] node {0,1} (q_1);
				\end{tikzpicture}
			\end{center}
		\item El lenguaje formado por la palabra $01$, o sea, $\{01\}$.
			\begin{center}
				\begin{tikzpicture}[>=stealth,auto]
					\node[state,initial] (q_0) {};
					\node[state] (q_1) [right=of q_0] {};
					\node[state, accepting] (q_2) [right=of q_1] {};
					\node[state] (e) [below=of q_1] {};
					\path[->]
					(q_0) edge node {0} (q_1)
					(q_0) edge [swap] node {1} (e)
					(q_1) edge node {0} (e)
					(q_1) edge node {1} (q_2)
					(q_2) edge node {0,1} (e)
					(e) edge [loop below] node {0,1} (e);
				\end{tikzpicture}
			\end{center}
		\item El lenguaje $\{11,00\}$.
			\begin{center}
				\begin{tikzpicture}[>=stealth,auto]
					\node[state,initial] (q_0) {};
					\node[state] (q_1) [above=of q_0] {};
					\node[state, accepting] (q_2) [right=of q_1] {};
					\node[state] (q_3) [below=of q_0] {};
					\node[state, accepting] (q_4) [right=of q_3] {};
					\node[state] (e) [right=of q_0] {};
					\path[->]
					(q_0) edge node {0} (q_1)
					(q_0) edge [swap] node {1} (q_3)
					(q_1) edge node {0} (q_2)
					(q_1) edge [swap] node {1} (e)
					(q_2) edge node {0,1} (e)
					(q_3) edge node {0} (e)
					(q_3) edge [swap] node {1} (q_4)
					(q_4) edge [swap] node {0,1} (e)
					(e) edge [loop right] node {0,1} (e);
				\end{tikzpicture}
			\end{center}
		\item El lenguaje formado por sucesiones de la subcadena $'01'$ incluyendo la cadena vacía, o sea, $\{\varepsilon, 01, 0101, 010101,\dots\}$.
			\begin{center}
				\begin{tikzpicture}[>=stealth,auto]
					\node[state,initial, accepting] (q_0) {};
					\node[state] (q_1) [right=of e] {};
					\node[state] (e) [right=of q_0] {};
					\path[->]
					(q_0) edge [bend left=50] node {0} (q_1)
					(q_0) edge node {1} (e)
					(q_1) edge [swap] node {0} (e)
					(q_1) edge [bend left=75] node {1} (q_0)
					(e) edge [loop below] node {0,1} (e);
				\end{tikzpicture}
			\end{center}
		\item El lenguaje formado por las cadenas donde el número de $1's$ es divisible por 3.
			\begin{center}
				\begin{tikzpicture}[>=stealth,auto]
					\node[state,initial,accepting] (q_0) {};
					\node[state] (q_1) [right=of q_0] {};
					\node[state] (q_2) [right=of q_1] {};
					\path[->]
					(q_0) edge [loop above] node {0} (q_0)
					(q_0) edge node {1} (q_1)
					(q_1) edge [loop above] node {0} (q_1)
					(q_1) edge node {1} (q_2)
					(q_2) edge [loop above] node {0} (q_2)
					(q_2) edge [bend left=40] node {1} (q_0);
				\end{tikzpicture}
			\end{center}
	\end{enumerate}
	\item Dado el siguiente autómata $M$, describir el lenguaje aceptado por dicho autómata:
	\begin{center}
		\begin{tikzpicture}[>=stealth,auto]
			\node[state,initial] (q_0) {};
			\node[state] (q_1) [right=of q_0] {};
			\node[state,accepting](q_2) [right=of q_1] {};
			\path[->]
			(q_0) edge  [loop above] node {b} (q_0)
			(q_0) edge  node {a} (q_1)
			(q_1) edge  [loop above] node {b} (q_1)
			(q_1) edge  [bend left] node {a} (q_2)
			(q_2) edge  [loop above] node {b} (q_2)
			(q_2) edge  [bend left] node {a} (q_1);
		\end{tikzpicture}\\
	\end{center}
	Genera el lenguaje $L=\{b^ia(b^ja)^{2k+1}b^l: i,j,k,l \in \mathbb{N}\}$ con $A=\{a,b\}$.
\end{enumerate}
\end{document}
