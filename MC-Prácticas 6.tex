%%
% Plantilla de Memoria
% Modificación de una plantilla de Latex de Nicolas Diaz para adaptarla 
% al castellano y a las necesidades de escribir informática y matemáticas.
%
% Editada por: Mario Román
%
% License:
% CC BY-NC-SA 3.0 (http://creativecommons.org/licenses/by-nc-sa/3.0/)
%%

%%%%%%%%%%%%%%%%%%%%%
% Thin Sectioned Essay
% LaTeX Template
% Version 1.0 (3/8/13)
%
% This template has been downloaded from:
% http://www.LaTeXTemplates.com
%
% Original Author:
% Nicolas Diaz (nsdiaz@uc.cl) with extensive modifications by:
% Vel (vel@latextemplates.com)
%
% License:
% CC BY-NC-SA 3.0 (http://creativecommons.org/licenses/by-nc-sa/3.0/)
%
%%%%%%%%%%%%%%%%%%%%%

%----------------------------------------------------------------------------------------
%	PAQUETES Y CONFIGURACIÓN DEL DOCUMENTO
%----------------------------------------------------------------------------------------

%% Configuración del papel.
% microtype: Tipografía.
% mathpazo: Usa la fuente Palatino.
\documentclass[a4paper, 11pt]{article}
\usepackage[protrusion=true,expansion=true]{microtype}
\usepackage{mathpazo}

% Indentación de párrafos para Palatino
\setlength{\parindent}{0pt}
  \parskip=8pt
\linespread{1.05} % Change line spacing here, Palatino benefits from a slight increase by default


%% Castellano.
% noquoting: Permite uso de comillas no españolas.
% lcroman: Permite la enumeración con numerales romanos en minúscula.
% fontenc: Usa la fuente completa para que pueda copiarse correctamente del pdf.
\usepackage[spanish,es-noquoting,es-lcroman]{babel}
\usepackage[utf8]{inputenc}
\usepackage[T1]{fontenc}
\selectlanguage{spanish}


%% Gráficos
\usepackage{graphicx} % Required for including pictures
\usepackage{wrapfig} % Allows in-line images
\usepackage[usenames,dvipsnames]{color} % Coloring code


%% Matemáticas
\usepackage{amsmath}


%% Bibliografía
\makeatletter
\renewcommand\@biblabel[1]{\textbf{#1.}} % Change the square brackets for each bibliography item from '[1]' to '1.'
\renewcommand{\@listI}{\itemsep=0pt} % Reduce the space between items in the itemize and enumerate environments and the bibliography

\usepackage{qtree}
\usepackage{tikz}
\usetikzlibrary{automata,positioning}

%----------------------------------------------------------------------------------------
%	TÍTULO
%----------------------------------------------------------------------------------------
% Configuraciones para el título.
% El título no debe editarse aquí.
\renewcommand{\maketitle}{
  \begin{flushright} % Right align
  
  {\LARGE\@title} % Increase the font size of the title
  
  \vspace{50pt} % Some vertical space between the title and author name
  
  {\large\@author} % Author name
  \\\@date % Date
  \vspace{40pt} % Some vertical space between the author block and abstract
  \end{flushright}
}

% Título
\title{\textbf{Modelos de Computación}\\ % Title
Relación de Prácticas 6} % Subtitle

\author{\textsc{Óscar Bermúdez Garrido} % Author
\\{\textit{Universidad de Granada}}} % Institution

\date{\today} % Date



%----------------------------------------------------------------------------------------
%	DOCUMENTO
%----------------------------------------------------------------------------------------

\begin{document}

\maketitle % Print the title section

\begin{enumerate}
	\item Dar un autómata con pila que acepte las cadenas del siguiente lenguaje por el
	criterio de pila vacía:
	
	\begin{itemize}
		\item $L = \{a^ib^jc^kd^l / (i=l) \vee (j=k)\}$
	\end{itemize}

	Primero, construimos una gramática que represente ese lenguaje, por ejemplo:
	\begin{center}
		\begin{tabular}{l l}
			$S \rightarrow S_1$ & $S \rightarrow S_4$ \\
			$S_1 \rightarrow aS_1d$ | $S_2$ & $S_4 \rightarrow aS_4$ | $S_5S_6$ \\
			$S_2 \rightarrow bS_2$ | $S_3$ & $S_5 \rightarrow bS_5c$ | $\varepsilon$ \\
			$S_3 \rightarrow cS_3$ | $\varepsilon$ & $S_6 \rightarrow dS_6$ | $\varepsilon$
		\end{tabular}
	\end{center}
	
	Ahora, pasamos al autómata con pila asociado:
	
	$M=(\{ q \}, \{ a, b, c, d \}, \{ a, b, c, d, S, S_1, S_2, S_3, S_4, S_5, S_6\}, \delta, q, S, \emptyset)$

	\begin{center}
		\begin{tabular}{l l}
			$\delta(q, \varepsilon, S)=\{ (q, S_1), (q, S_4) \}$ & $\delta(q, \varepsilon, S_1)=\{ (q, aS_1d), (q, S_2) \}$ \\
			$\delta(q, \varepsilon, S_2)=\{ (q, bS_2), (q, S_3) \}$ & $\delta(q, \varepsilon, S_3)=\{ (q, cS_3), (q, \varepsilon) \}$ \\
			$\delta(q, \varepsilon, S_4)=\{ (q, aS_4), (q, S_5S_6) \}$ & $\delta(q, \varepsilon, S_5)=\{ (q, bS_5c), (q, \varepsilon) \}$ \\
			$\delta(q, \varepsilon, S_6)=\{ (q, dS_6), (q, \varepsilon) \}$ & \\
			$\delta(q, a, a)=\{ (q, \varepsilon) \}$ & $\delta(q, b, b)=\{ (q, \varepsilon) \}$ \\
			$\delta(q, c, c)=\{ (q, \varepsilon) \}$ & $\delta(q, d, d)=\{ (q, \varepsilon) \}$ \\
		\end{tabular}
	\end{center}
	
	\item Dar un autómata con pila determinista que acepte las cadenas definidas sobre el
	alfabeto $A$ de los siguientes lenguajes por el criterio de pila vacía, si no es posible
	encontrarlo por ese criterio entonces usar el criterio de estados finales:
	
	\begin{enumerate}
		\item $L_1 = \{ 0^i1^j2^k3^m / i,j,k \geq 0, m=i+j+k \}$ con $A = \{ 0, 1, 2, 3 \}$
		
		$M=(\{ q_\varepsilon, q_0, q_1, q_2, q_3 \}, \{ 0, 1, 2, 3 \}, \{ R, X \}, \delta, q_\varepsilon, R, \{q_\varepsilon\})$
		\begin{center}
			\begin{tabular}{l l l}
				$\delta(q_\varepsilon, 0, R)=\{ (q_0, XR) \}$ & $\delta(q_\varepsilon, 1, R)=\{ (q_1, XR) \}$ & $\delta(q_\varepsilon, 2, R)=\{ (q_2, XR) \}$ \\
				$\delta(q_0, 0, X)=\{ (q_0, XX) \}$ & $\delta(q_0, 1, X)=\{ (q_1, XX) \}$ & $\delta(q_0, 2, X)=\{ (q_2, XX) \}$ \\
				$\delta(q_0, 3, X)=\{ (q_3, \varepsilon) \}$ & & \\
				$\delta(q_1, 1, X)=\{ (q_1, XX) \}$ & $\delta(q_1, 2, X)=\{ (q_2, XX) \}$ & $\delta(q_1, 3, X)=\{ (q_3, \varepsilon) \}$ \\
				$\delta(q_2, 2, X)=\{ (q_2, XX) \}$ & $\delta(q_2, 3, X)=\{ (q_3, \varepsilon) \}$ & \\
				$\delta(q_3, 3, X)=\{ (q_3, \varepsilon) \}$ & $\delta(q_3, \varepsilon, R)=\{ (q_\varepsilon, \varepsilon) \}$ & \\
			\end{tabular}
		\end{center}
			
		\item $L_2 = \{ 0^i1^j2^k3^m4 / i,j,k \geq 0, m=i+j+k \}$ con $A = \{ 0, 1, 2, 3, 4 \}$
		
		$M=(\{ q_0, q_1, q_2, q_3 \}, \{ 0, 1, 2, 3, 4 \}, \{ R, X\}, \delta, q_0, R, \emptyset)$
		\begin{center}
			\begin{tabular}{l l l}
				$\delta(q_0, 0, R)=\{ (q_0, XR) \}$ & $\delta(q_0, 0, X)=\{ (q_0, XX) \}$ & $\delta(q_0, 1, R)=\{ (q_1, XR) \}$ \\
				$\delta(q_0, 1, X)=\{ (q_1, XX) \}$ & $\delta(q_0, 2, R)=\{ (q_2, XR) \}$ &  $\delta(q_0, 2, X)=\{ (q_2, XX) \}$ \\
				$\delta(q_0, 3, X)=\{ (q_3, \varepsilon) \}$ & $\delta(q_0, 4, R)=\{ (q_0, \varepsilon) \}$ & \\
				$\delta(q_1, 1, X)=\{ (q_1, XX) \}$ & $\delta(q_1, 2, X)=\{ (q_2, XX) \}$ & $\delta(q_1, 3, X)=\{ (q_3, \varepsilon) \}$ \\
				$\delta(q_2, 2, X)=\{ (q_2, XX) \}$ & $\delta(q_2, 3, X)=\{ (q_3, \varepsilon) \}$ & \\
				$\delta(q_3, 3, X)=\{ (q_3, \varepsilon) \}$ & & \\
				$\delta(q_3, 4, R)=\{ (q_3, \varepsilon) \}$ & & \\
			\end{tabular}
		\end{center}
	\end{enumerate}
	
	Si en alguno de los lenguajes anteriores no ha sido posible encontrar un autómata con pila
	determinista por el criterio de pila vacía, entonces justifica por qué no ha sido posible.
	
	En el primer lenguaje, no fue posible encontrar un autómata con pila determinista por el
	criterio de pila vacía porque como el símbolo inicial es siempre $"R"$ y el lenguaje acepta
	$\varepsilon$ y otras palabras no nulas, tenemos que $\delta(q_0,a,R) \neq \emptyset$ con
	$a \in A$ y $\delta(q_0,\varepsilon,R) \neq \emptyset$, lo que contradice la definición de
	autómata con pila determinista.
\end{enumerate}

\end{document}
